%%%%%%%%%%%%%%%%%%%%%%%%%%%%%%%%%%%%%%%%%
% Medium Length Graduate Curriculum Vitae
% LaTeX Template
% Version 1.1 (9/12/12)
%
% This template has been downloaded from:
% http://www.LaTeXTemplates.com
%
% Original author:
% Rensselaer Polytechnic Institute (http://www.rpi.edu/dept/arc/training/latex/resumes/)
%
% Important note:
% This template requires the res.cls file to be in the same directory as the
% .tex file. The res.cls file provides the resume style used for structuring the
% document.
%
%%%%%%%%%%%%%%%%%%%%%%%%%%%%%%%%%%%%%%%%%

%----------------------------------------------------------------------------------------
%	PACKAGES AND OTHER DOCUMENT CONFIGURATIONS
%----------------------------------------------------------------------------------------

\documentclass[margin, 10pt]{res} % Use the res.cls style, the font size can be changed to 11pt or 12pt here

\usepackage{helvet} % Default font is the helvetica postscript font
%\usepackage{newcent} % To change the default font to the new century schoolbook postscript font uncomment this line and comment the one above
\usepackage{cmap}
\usepackage{mathtext}
\usepackage[T2A]{fontenc}
\usepackage[utf8]{inputenc}
\usepackage[english,russian]{babel}
\usepackage{hyperref}
\usepackage{xcolor}

% \definecolor{linkcolor}{HTML}{799B03}
% \definecolor{urlcolor}{HTML}{799B03}
\hypersetup{pdfstartview=FitH, colorlinks=true}

\setlength{\textwidth}{5.1in} % Text width of the document

\begin{document}

%----------------------------------------------------------------------------------------
%	NAME AND ADDRESS SECTION
%----------------------------------------------------------------------------------------

\moveleft.5\hoffset\centerline{\large\bf Андрей Одинцов} % Your name at the top
 
\moveleft\hoffset\vbox{\hrule width\resumewidth height 1pt}\smallskip % Horizontal line after name; adjust line thickness by changing the '1pt'
 
\moveleft.5\hoffset\centerline{tg: \href{https://teleg.run/forestryks}{@forestryks}}
\moveleft.5\hoffset\centerline{+7 (953) 144-30-97}
\moveleft.5\hoffset\centerline{forestryks1@gmail.com} % Your address
\moveleft.5\hoffset\centerline{\href{https://github.com/forestryks}{github}}


%----------------------------------------------------------------------------------------

\begin{resume}

\newsectionwidth{100pt}

\section{ОБРАЗОВАНИЕ}

Учусь на первом курсе в НИУ <<ВШЭ>> на ПМИ ФКН.

\section{ОЛИМПИАДЫ}

\begin{itemize}
	\item Победитель всероссийской олимпиады школьников по информатике 2019
	\item Победитель открытой олимпиады по программированию 2020
	\item ВКОШП 2019 - 4 место
	\item Серебрянная медаль IATI 2019
	\item Золотая медаль APIO 2019
	\item Профиль codeforces: \href{https://codeforces.com/profile/forestryks}{forestryks}
\end{itemize}

\section{ПРОЕКТЫ}

\begin{itemize}
	\item Патч к ядру Linux, добавляющий безопасный режим запуска процессов с настраиваемыми ограничениями на системые вызовы и ресурсы системы, а также экспорт из ядра данных по использованию памяти процессом. Часть написана вручную на си, часть генерируется автоматически скриптом на питоне. (\href{https://github.com/forestryks/kJudge}{github})
	\item Контейнеризация с максимально оптимизированным временем запуска для использования в тестирующих системах. Написан на C++, использует linux namespaces и cgroups. (\href{https://github.com/forestryks/libsbox}{github})
	\item Небольшая библиотека на C++ для генерации json serializers/deserializers для структур используя макросы и шаблоны. (\href{https://github.com/forestryks/json-model}{github})
	\item Система для проверки на списывание в контестах по программированию. Бекенд написан на rust. Репоз приватный, могу показать при необходимости.
	\item С этого года преподаю алгоритмы школьникам в Tinkoff Generation.
\end{itemize}

\section{НАВЫКИ}
\begin{itemize}
	\item Пишу на C++, Rust и Python. C++~--- первый язык программирования, опыт около 5 лет. Опыта написания кода на питоне меньше, хотя пишу на нем с некоторой периодичностью последние 4 года. В основном использую для автоматизации чего-либо. На расте стал писать в течение последнего года. Также имею небольшой опыт программирования на golang.
	\item Работаю в Linux, активно пользуюсь терминалом, понимаю основы синтаксиса bash (for написать смогу, с dfs с ходу не справлюсь). В частности, гитом пользоваться умею.
	\item Как-то приходилось с нуля настраивать zabbix и ansible.
	\item Хорошо разбираюсь во внутреннем устройсте системы, так как делал проекты в этой области, в частности, патч к ядру.
	\item Имею представление о принципах работы докера, и как им пользоваться.
\end{itemize}

\end{resume}
\end{document}
